%
% API Documentation for DataEnvironment
% Module System.SystemGraphOperations
%
% Generated by epydoc 3.0.1
% [Tue Mar 31 17:17:20 2009]
%

%%%%%%%%%%%%%%%%%%%%%%%%%%%%%%%%%%%%%%%%%%%%%%%%%%%%%%%%%%%%%%%%%%%%%%%%%%%
%%                          Module Description                           %%
%%%%%%%%%%%%%%%%%%%%%%%%%%%%%%%%%%%%%%%%%%%%%%%%%%%%%%%%%%%%%%%%%%%%%%%%%%%

    \index{System \textit{(package)}!System.SystemGraphOperations \textit{(module)}|(}
\section{Module System.SystemGraphOperations}

    \label{System:SystemGraphOperations}
Rountines for constructing meta-data graphs describing the link structure 
local to a give path in the Data Enviroment.

The primary use of the graphs constructed by this module are in the 
Graphical Browser .


%%%%%%%%%%%%%%%%%%%%%%%%%%%%%%%%%%%%%%%%%%%%%%%%%%%%%%%%%%%%%%%%%%%%%%%%%%%
%%                               Functions                               %%
%%%%%%%%%%%%%%%%%%%%%%%%%%%%%%%%%%%%%%%%%%%%%%%%%%%%%%%%%%%%%%%%%%%%%%%%%%%

  \subsection{Functions}

    \label{System:SystemGraphOperations:MakeLocalLinkList}
    \index{System \textit{(package)}!System.SystemGraphOperations \textit{(module)}!System.SystemGraphOperations.MakeLocalLinkList \textit{(function)}}

    \vspace{0.5ex}

\hspace{.8\funcindent}\begin{boxedminipage}{\funcwidth}

    \raggedright \textbf{MakeLocalLinkList}(\textit{Path}, \textit{depends\_on}={\tt ('../System/StoredLinks/StoredImpliedLinks','../System/St\texttt{...}}, \textit{creates}={\tt ('../System/MetaData',)})

    \vspace{-1.5ex}

    \rule{\textwidth}{0.5\fboxrule}
\setlength{\parskip}{2ex}
\begin{alltt}

Given, Path, a path string describing a location in the Data Environment, 
get the 2-neighborhood graph of the linklist local to that path.

It uses the linklist loaded from the Live Modules. 

This function outputs the result, as a numpy sub- record array of 
the Linklist, to a pickled file whose path is given by:

        MetaPathDir(Path) + 'LocalLinkList.pickle'

This path is in in the meta-data directory associated with Path.

It also outputs the file MetaPathDir(Path) + 'LiveModuleList.pickle', 
which is used to cache this computation for future recomputations of it. 

RETURNS:
        LinkPath = path where the locallink file was created
        (i.e. MetaPathDir(Path) + 'LocalLinkList.pickle')
\end{alltt}

\setlength{\parskip}{1ex}
    \end{boxedminipage}

    \label{System:SystemGraphOperations:MakeLocalLinkGraph}
    \index{System \textit{(package)}!System.SystemGraphOperations \textit{(module)}!System.SystemGraphOperations.MakeLocalLinkGraph \textit{(function)}}

    \vspace{0.5ex}

\hspace{.8\funcindent}\begin{boxedminipage}{\funcwidth}

    \raggedright \textbf{MakeLocalLinkGraph}(\textit{Path}, \textit{Mode}={\tt 'ColDir'}, \textit{ShowUses}={\tt 'No'}, \textit{ShowImplied}={\tt 'No'})

    \vspace{-1.5ex}

    \rule{\textwidth}{0.5\fboxrule}
\setlength{\parskip}{2ex}
\begin{alltt}

Takes a local link list as made by MakeLocalLinkList and turns it a 
graph .svg file.  (also creates an html-ized version of the local linkgraph list.

The basic strategy of this functions is:
        1) load the local link list created in a numpy pickle file 
                at the path Linkpath
        2) Cluster the nodes based on the mode:  
                -- if Mode = 'ColDir' then collapse nodes within directories
                -- if Mode = 'ColPro' the collapse nodes based on their being part of 
                protocols
                -- if Mode = 'All' then do no collapsing
        3) Create an html reprsentation of the link graph (with the collapsed
                clusters notated in it)
        4) Create a graph with the collapsed nodes as an object of the form 
                
                G = [Nodes,Edges, NodeAttributes,EdgeAttributes]
        
        5) convert the G object into .dot and then .svg files.
        

ARGUMENTS:
--Path : path to make graph of local link list for
--Mode : string, which "graph quotienting" mode to use
--ShowUses : boolean, whether to include uses links or not
--ShowImplied : boolean, whether to include implied links or not

RETURNS:        
[message,metapath,G,metapathhtml], where:
                
        message is an error message, if any (blank '' char if no error message)
        metapath = path where .svg file has been created
        
        G = [N,E,NAttrs,EAttrs]
        
        where N and E is a list of the graph nodes and edges,
        as a python list, and NAttrs and EAttrs are corresponding
        lists of node and edge attributes (for .dot graph format)
        
        metapathhtml = path where an html representation of the link 
        list has been stored
                        
\end{alltt}

\setlength{\parskip}{1ex}
    \end{boxedminipage}

    \label{System:SystemGraphOperations:MakeLinkListHtml}
    \index{System \textit{(package)}!System.SystemGraphOperations \textit{(module)}!System.SystemGraphOperations.MakeLinkListHtml \textit{(function)}}

    \vspace{0.5ex}

\hspace{.8\funcindent}\begin{boxedminipage}{\funcwidth}

    \raggedright \textbf{MakeLinkListHtml}(\textit{LinkList}, \textit{ClusterTags}, \textit{inpath}, \textit{outpath}, \textit{Mode}, \textit{ShowUses}, \textit{ShowImplied})

    \vspace{-1.5ex}

    \rule{\textwidth}{0.5\fboxrule}
\setlength{\parskip}{2ex}
\begin{alltt}

takes a data dependency linklist and a list of cluster tags, and 
outputs an html representation of the linklist file (basically as
a clickable html table) 

ARGUMENTS:
--LinkList = the linklist to reresent, a numpy record array
--ClusterTags = dictionary of clusters (output for example of
        the GetClusterTagDict function)
--inpath = path that the LinkList is a representation of
--outpath = place to store the resulting Html file
--Mode = Name of the cluster collapse mode
--ShowUses, ShowImplied are as in MakeLocalLinkGraph
        
returns:
        nothing
                
\end{alltt}

\setlength{\parskip}{1ex}
    \end{boxedminipage}

    \label{System:SystemGraphOperations:LabeledGraphFromLinks}
    \index{System \textit{(package)}!System.SystemGraphOperations \textit{(module)}!System.SystemGraphOperations.LabeledGraphFromLinks \textit{(function)}}

    \vspace{0.5ex}

\hspace{.8\funcindent}\begin{boxedminipage}{\funcwidth}

    \raggedright \textbf{LabeledGraphFromLinks}(\textit{LinkList}, \textit{Path}, \textit{ClusterTags}={\tt None}, \textit{Mode}={\tt 'ColDir'}, \textit{ShowUses}={\tt 'No'}, \textit{ShowImplied}={\tt 'No'})

    \vspace{-1.5ex}

    \rule{\textwidth}{0.5\fboxrule}
\setlength{\parskip}{2ex}
\begin{alltt}

Produces a graphical representation for use in producing a
.dot file representation of a Linklist{\textgreater}

ARGUMENTS: as is MakeLocalLinkGraph

Returns:
        object G = [Nodes,Edges,NodeAttrs,EdgeAttrs]
where
        Nodes is a python list of node names
        Edges is a python list of pairs of node names, representing node edges
        NodeAttrs is a dictionary of node attributes where:
                -- the keys are node names
                -- the value on a key 'n' is a dictionary of .dot format 
                key-value attribute pairs for node 'n', e.g. 
                \{'color':'green','shape':'box', .... \} 
        EdgeAttrs is a dictionary of edge  attributes, where:
                -- the keys are edge pairs
                -- the value on a key 'e' is a dictionary of .dot format 
                key-values attribute pairs for edge 'e', e.g. 
                \{'color':'green','shape':'box', .... \} 
\end{alltt}

\setlength{\parskip}{1ex}
    \end{boxedminipage}

    \label{System:SystemGraphOperations:EdgePropertiesSelector}
    \index{System \textit{(package)}!System.SystemGraphOperations \textit{(module)}!System.SystemGraphOperations.EdgePropertiesSelector \textit{(function)}}

    \vspace{0.5ex}

\hspace{.8\funcindent}\begin{boxedminipage}{\funcwidth}

    \raggedright \textbf{EdgePropertiesSelector}(\textit{e}, \textit{EdgeInfo})

    \vspace{-1.5ex}

    \rule{\textwidth}{0.5\fboxrule}
\setlength{\parskip}{2ex}
    Technical dependency used in LabeledGraphFromLinks

\setlength{\parskip}{1ex}
    \end{boxedminipage}

    \label{System:SystemGraphOperations:EdgeTypeDeterminer}
    \index{System \textit{(package)}!System.SystemGraphOperations \textit{(module)}!System.SystemGraphOperations.EdgeTypeDeterminer \textit{(function)}}

    \vspace{0.5ex}

\hspace{.8\funcindent}\begin{boxedminipage}{\funcwidth}

    \raggedright \textbf{EdgeTypeDeterminer}(\textit{e}, \textit{EdgeInfo})

    \vspace{-1.5ex}

    \rule{\textwidth}{0.5\fboxrule}
\setlength{\parskip}{2ex}
    Technical dependency used in LabeledGraphFromLinks

\setlength{\parskip}{1ex}
    \end{boxedminipage}

    \label{System:SystemGraphOperations:NodePropertiesSelector}
    \index{System \textit{(package)}!System.SystemGraphOperations \textit{(module)}!System.SystemGraphOperations.NodePropertiesSelector \textit{(function)}}

    \vspace{0.5ex}

\hspace{.8\funcindent}\begin{boxedminipage}{\funcwidth}

    \raggedright \textbf{NodePropertiesSelector}(\textit{n}, \textit{Path}, \textit{ClusterDict}, \textit{NodeInfo}, \textit{Mode}, \textit{ShowUses}, \textit{ShowImplied})

    \vspace{-1.5ex}

    \rule{\textwidth}{0.5\fboxrule}
\setlength{\parskip}{2ex}
    Technical dependency used in LabeledGraphFromLinks

\setlength{\parskip}{1ex}
    \end{boxedminipage}

    \label{System:SystemGraphOperations:NodeTypeDeterminer}
    \index{System \textit{(package)}!System.SystemGraphOperations \textit{(module)}!System.SystemGraphOperations.NodeTypeDeterminer \textit{(function)}}

    \vspace{0.5ex}

\hspace{.8\funcindent}\begin{boxedminipage}{\funcwidth}

    \raggedright \textbf{NodeTypeDeterminer}(\textit{n}, \textit{NodeInfo}, \textit{Mode}, \textit{ShowUses}, \textit{ShowImplied})

    \vspace{-1.5ex}

    \rule{\textwidth}{0.5\fboxrule}
\setlength{\parskip}{2ex}
    Technical dependency used in LabeledGraphFromLinks

\setlength{\parskip}{1ex}
    \end{boxedminipage}

    \label{System:SystemGraphOperations:DeleteLinkGraphs}
    \index{System \textit{(package)}!System.SystemGraphOperations \textit{(module)}!System.SystemGraphOperations.DeleteLinkGraphs \textit{(function)}}

    \vspace{0.5ex}

\hspace{.8\funcindent}\begin{boxedminipage}{\funcwidth}

    \raggedright \textbf{DeleteLinkGraphs}()

\setlength{\parskip}{2ex}
\setlength{\parskip}{1ex}
    \end{boxedminipage}

    \label{System:SystemGraphOperations:DeleteLocalLinkLists}
    \index{System \textit{(package)}!System.SystemGraphOperations \textit{(module)}!System.SystemGraphOperations.DeleteLocalLinkLists \textit{(function)}}

    \vspace{0.5ex}

\hspace{.8\funcindent}\begin{boxedminipage}{\funcwidth}

    \raggedright \textbf{DeleteLocalLinkLists}()

\setlength{\parskip}{2ex}
\setlength{\parskip}{1ex}
    \end{boxedminipage}

    \label{System:SystemGraphOperations:inverse}
    \index{System \textit{(package)}!System.SystemGraphOperations \textit{(module)}!System.SystemGraphOperations.inverse \textit{(function)}}

    \vspace{0.5ex}

\hspace{.8\funcindent}\begin{boxedminipage}{\funcwidth}

    \raggedright \textbf{inverse}(\textit{S})

    \vspace{-1.5ex}

    \rule{\textwidth}{0.5\fboxrule}
\setlength{\parskip}{2ex}
    inverts permutation described a numpy array

\setlength{\parskip}{1ex}
    \end{boxedminipage}

    \label{System:SystemGraphOperations:GetEQ}
    \index{System \textit{(package)}!System.SystemGraphOperations \textit{(module)}!System.SystemGraphOperations.GetEQ \textit{(function)}}

    \vspace{0.5ex}

\hspace{.8\funcindent}\begin{boxedminipage}{\funcwidth}

    \raggedright \textbf{GetEQ}(\textit{X}, \textit{Y}, \textit{taglist})

\setlength{\parskip}{2ex}
\setlength{\parskip}{1ex}
    \end{boxedminipage}

    \label{System:SystemGraphOperations:ProcessTwoDicts}
    \index{System \textit{(package)}!System.SystemGraphOperations \textit{(module)}!System.SystemGraphOperations.ProcessTwoDicts \textit{(function)}}

    \vspace{0.5ex}

\hspace{.8\funcindent}\begin{boxedminipage}{\funcwidth}

    \raggedright \textbf{ProcessTwoDicts}(\textit{A}, \textit{B})

    \vspace{-1.5ex}

    \rule{\textwidth}{0.5\fboxrule}
\setlength{\parskip}{2ex}
    Technical dependency used in GetClusterTagDict

\setlength{\parskip}{1ex}
    \end{boxedminipage}

    \label{System:SystemGraphOperations:MaximalCP}
    \index{System \textit{(package)}!System.SystemGraphOperations \textit{(module)}!System.SystemGraphOperations.MaximalCP \textit{(function)}}

    \vspace{0.5ex}

\hspace{.8\funcindent}\begin{boxedminipage}{\funcwidth}

    \raggedright \textbf{MaximalCP}(\textit{P})

    \vspace{-1.5ex}

    \rule{\textwidth}{0.5\fboxrule}
\setlength{\parskip}{2ex}
    Technical dependency used in LabeledGraphFromLinks

\setlength{\parskip}{1ex}
    \end{boxedminipage}

    \label{System:SystemGraphOperations:SPathAlong}
    \index{System \textit{(package)}!System.SystemGraphOperations \textit{(module)}!System.SystemGraphOperations.SPathAlong \textit{(function)}}

    \vspace{0.5ex}

\hspace{.8\funcindent}\begin{boxedminipage}{\funcwidth}

    \raggedright \textbf{SPathAlong}(\textit{p1}, \textit{p2})

    \vspace{-1.5ex}

    \rule{\textwidth}{0.5\fboxrule}
\setlength{\parskip}{2ex}
    Technical dependency used in GetClusterTagDict

\setlength{\parskip}{1ex}
    \end{boxedminipage}

    \label{System:SystemGraphOperations:GetClusterTagDict}
    \index{System \textit{(package)}!System.SystemGraphOperations \textit{(module)}!System.SystemGraphOperations.GetClusterTagDict \textit{(function)}}

    \vspace{0.5ex}

\hspace{.8\funcindent}\begin{boxedminipage}{\funcwidth}

    \raggedright \textbf{GetClusterTagDict}(\textit{Path}, \textit{LinkList}, \textit{Mode})

    \vspace{-1.5ex}

    \rule{\textwidth}{0.5\fboxrule}
\setlength{\parskip}{2ex}
\begin{alltt}

Given a  LinkList and a mode for collapsing the Linklist, 
produce a dictionary of node cluster tags. 

ARGUMENTS:
        Path -- path that the LinkList corresponds to
        LinkList -- list of Links as a numy record array
        Mode -- Mode by which to cluster the link nodes
        
RETURNS:
        ClusterDict, a dictionary in which:
        -- the keys are (some of the) link sources or targets in the 
                LinkList (e.g. potential nodes in the graph of the linklist)
        -- value on a key is a "cluster name " which represents a set 
        of nodes that will collapsed together
        
The "mode" determines which collapsing algorithm should be used.  
        
\end{alltt}

\setlength{\parskip}{1ex}
    \end{boxedminipage}

    \label{System:SystemGraphOperations:MetaPathDir}
    \index{System \textit{(package)}!System.SystemGraphOperations \textit{(module)}!System.SystemGraphOperations.MetaPathDir \textit{(function)}}

    \vspace{0.5ex}

\hspace{.8\funcindent}\begin{boxedminipage}{\funcwidth}

    \raggedright \textbf{MetaPathDir}(\textit{Path})

\setlength{\parskip}{2ex}
\setlength{\parskip}{1ex}
    \end{boxedminipage}

    \label{System:SystemGraphOperations:WriteOutGraphDot}
    \index{System \textit{(package)}!System.SystemGraphOperations \textit{(module)}!System.SystemGraphOperations.WriteOutGraphDot \textit{(function)}}

    \vspace{0.5ex}

\hspace{.8\funcindent}\begin{boxedminipage}{\funcwidth}

    \raggedright \textbf{WriteOutGraphDot}(\textit{G}, \textit{outpath})

    \vspace{-1.5ex}

    \rule{\textwidth}{0.5\fboxrule}
\setlength{\parskip}{2ex}
\begin{alltt}

Writes out graph in the [N,E,Nattrs,Eattrs] format to a .dot file

ARGUMENTS:

object G = [Nodes,Edges,NodeAttrs,EdgeAttrs], where
        Nodes is a python list of node names
        Edges is a python list of pairs of node names, representing node edges
        NodeAttrs is a dictionary of node attributes where:
                -- the keys are node names
                -- the value on a key 'n' is a dictionary of .dot format 
                key-value attribute pairs for node 'n', e.g.
                \{'color':'green','shape':'box', .... \} 
        EdgeAttrs is a dictionary of edge  attributes, where:
                -- the keys are edge pairs
                -- the value on a key 'e' is a dictionary of .dot format 
                key-values attribute pairs for edge 'e', e.g. 
                \{'color':'green','shape':'box', .... \} 

        
outpath -- path where .dot will be put
        
RETURNS:
        nothing
\end{alltt}

\setlength{\parskip}{1ex}
    \end{boxedminipage}

    \label{System:SystemGraphOperations:LabelFunc}
    \index{System \textit{(package)}!System.SystemGraphOperations \textit{(module)}!System.SystemGraphOperations.LabelFunc \textit{(function)}}

    \vspace{0.5ex}

\hspace{.8\funcindent}\begin{boxedminipage}{\funcwidth}

    \raggedright \textbf{LabelFunc}(\textit{n}, \textit{Path}, \textit{Other}, \textit{IsCluster})

    \vspace{-1.5ex}

    \rule{\textwidth}{0.5\fboxrule}
\setlength{\parskip}{2ex}
    Technical dependency of LabeledGraphFromLinks

\setlength{\parskip}{1ex}
    \end{boxedminipage}

    \label{System:SystemGraphOperations:IsProtocolPath}
    \index{System \textit{(package)}!System.SystemGraphOperations \textit{(module)}!System.SystemGraphOperations.IsProtocolPath \textit{(function)}}

    \vspace{0.5ex}

\hspace{.8\funcindent}\begin{boxedminipage}{\funcwidth}

    \raggedright \textbf{IsProtocolPath}(\textit{s})

    \vspace{-1.5ex}

    \rule{\textwidth}{0.5\fboxrule}
\setlength{\parskip}{2ex}
    Technical dependency of LabelFunc

\setlength{\parskip}{1ex}
    \end{boxedminipage}

    \label{System:SystemGraphOperations:MostCommonValue}
    \index{System \textit{(package)}!System.SystemGraphOperations \textit{(module)}!System.SystemGraphOperations.MostCommonValue \textit{(function)}}

    \vspace{0.5ex}

\hspace{.8\funcindent}\begin{boxedminipage}{\funcwidth}

    \raggedright \textbf{MostCommonValue}(\textit{D})

    \vspace{-1.5ex}

    \rule{\textwidth}{0.5\fboxrule}
\setlength{\parskip}{2ex}
    Given numpy array D, determines most common value D

\setlength{\parskip}{1ex}
    \end{boxedminipage}

    \label{System:SystemGraphOperations:OutsideFile}
    \index{System \textit{(package)}!System.SystemGraphOperations \textit{(module)}!System.SystemGraphOperations.OutsideFile \textit{(function)}}

    \vspace{0.5ex}

\hspace{.8\funcindent}\begin{boxedminipage}{\funcwidth}

    \raggedright \textbf{OutsideFile}(\textit{path})

    \vspace{-1.5ex}

    \rule{\textwidth}{0.5\fboxrule}
\setlength{\parskip}{2ex}
    Determines for purposes of graph coloring whether a file is an 
    "outside" file -- right now, the only kind of "outside" file that is 
    recognized are web files. Perhaps direct connections to servers and 
    databases should be added here for various formats eventually.

\setlength{\parskip}{1ex}
    \end{boxedminipage}


%%%%%%%%%%%%%%%%%%%%%%%%%%%%%%%%%%%%%%%%%%%%%%%%%%%%%%%%%%%%%%%%%%%%%%%%%%%
%%                               Variables                               %%
%%%%%%%%%%%%%%%%%%%%%%%%%%%%%%%%%%%%%%%%%%%%%%%%%%%%%%%%%%%%%%%%%%%%%%%%%%%

  \subsection{Variables}

    \vspace{-1cm}
\hspace{\varindent}\begin{longtable}{|p{\varnamewidth}|p{\vardescrwidth}|l}
\cline{1-2}
\cline{1-2} \centering \textbf{Name} & \centering \textbf{Description}& \\
\cline{1-2}
\endhead\cline{1-2}\multicolumn{3}{r}{\small\textit{continued on next page}}\\\endfoot\cline{1-2}
\endlastfoot\raggedright G\-r\-a\-p\-h\-I\-s\-T\-o\-o\-L\-a\-r\-g\-e\- & \raggedright \textbf{Value:} 
{\tt 500}&\\
\cline{1-2}
\end{longtable}

    \index{System \textit{(package)}!System.SystemGraphOperations \textit{(module)}|)}
